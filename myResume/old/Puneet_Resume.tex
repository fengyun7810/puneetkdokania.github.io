% LaTeX file for resume 
% This file uses the resume document class (res.cls)

    \documentclass{res}
    \usepackage{color} 
    \usepackage{url}
%\usepackage{helvetica} % uses helvetica postscript font (download helvetica.sty)
    %\usepackage{newcent}   % uses new century schoolbook postscript font 
    \setlength{\textheight}{9.5in} % increase text height to fit on 1-page 

    \begin{document} 
\vspace{-20pt}
    \name{PUNEET KUMAR DOKANIA\\ \color{blue}\;\; http://cvn.ecp.fr/personnel/puneet/}     % the \\[12pt] adds a blank
    % line after name      

%    \address{\bf  OFFICE ADDRESS\\Center for Visual Computing (CVC)\\A 205, Batiment Annexe Dumas, Ecole Centrale de Paris\\Chatenay Malabry, France - 92290\\+33 (0) 611530395\\ \color{blue}puneetkdokania@gmail.com\\
%        \color{blue}http://cvn.ecp.fr/personnel/puneet/}
%        \address{\bf PERMANENT ADDRESS (India)\\ C/O - Sri B. K. Dokania\\WP 125 (B) \\  Pitampura, Delhi - 110088, India \\  +91 (0) 9818343286}
%
        \begin{resume}

\section{INTERESTS} Thoeretical and practical problems related to Machine Learning, Computer Vision, Approximation Algorithms and Inference in Graphical Models. I am interested in almost every critical problems related to the above mentioned fields. 

        \section{EDUCATION} 
{\bf Ecole Centrale de Paris and INRIA Scalay}\\ Center for Learning and Visual Computing\\ PhD Student : October 2012 - {\bf Present}\\ Supervisors: Prof. M. Pawan Kumar and Prof. Nikos Paragios \\  \\
{\bf Ecole Nationale Supérieure d'Informatique et Mathématiques Appliquées}, Grenoble, France  \\        
Master of Science in Informatics with specialization in Graphics, Vision and Robotics\\ Sept 2011 - July 2012   \\  
{\bf Delhi College of Engineering}, University of Delhi, India       \\   
Bachelor in Computer Engineering, 2005 - May 2009  \\        
First Class Degree          

\vspace{-5pt}
\section{EXPERIENCES}
\vspace{-0.1in}	
\begin{tabbing}
\hspace{2.3in}\= \hspace{2.6in}\= \kill % set up two tab positions
{\bf Masters Thesis} \>E-Motion Team, INRIA \>Oct 2011 - June 2012
        %\>Grenoble, France
		\end{tabbing}\vspace{-18pt}      % suppress blank line after tabbing
        Learning based approach for online lane change intention prediction. Resulted in a reputed international conference paper in this field.
        \begin{tabbing}
        \hspace{2.3in}\= \hspace{2.6in}\= \kill % set up two tab positions
{\bf Research Scientist} \>Advanced Systems Laboratory, India \> Dec 2009 to Aug 2011
        %\> Govt. of India
        \end{tabbing}\vspace{-18pt}	   % suppress blank line after tabbing
        Worked on the navigation of Unmanned Aerial Vehicle using EKF based integration of INS and GPS. Published one conference paper.
        \begin{tabbing}%
        \hspace{2.3in}\= \hspace{2.6in}\= \kill % set up two tab positions          
{\bf Research Assistant}  \>IIT Delhi, India \> June 2009 to Nov 2009

    \end{tabbing}\vspace{-18pt}
    Worked on the applications of Swarm Intelligence in the field of Robotics and Image Processing. Published two international journals and one international conference papers as the outcome of research conducted here.

    %\begin{tabbing}%
    %   \hspace{2.3in}\= \hspace{2.6in}\= \kill % set up two tab positions          
    %   {\bf Internship Student}  \>Indian Space Research Organisation \> Dec 2008 to Feb 2009\\
        %\>IIRS, Dehradun, India
        %                          
        %   \end{tabbing}\vspace{-20pt}
        %    Worked on the denoising of time series NDVI dataset using wavelet transformation. Along with this I worked on the dimensionality reduction of hyperspectral images using ICA.

        \begin{tabbing}%
        \hspace{2.3in}\= \hspace{2.6in}\= \kill % set up two tab positions          
{\bf Internship Student}  \>AI and Vision Lab, IISc India \> June 2007 to 3rd Aug 07
        %\>IISc Bangalore, India

        \end{tabbing}\vspace{-18pt}
        Worked on the depth perception of a scene using Stereovision and developed an autonomous corridor navigator using stereovision camera. Implemented a simple probabilistic approach for obstacle avoidance.

        \begin{tabbing}%
        \hspace{2.3in}\= \hspace{2.6in}\= \kill % set up two tab positions          
{\bf Technical Head and Cofounder}  \>IGVC Team, DCE, India\> March 2007 to Nov 2008
        %\>Delhi College of Engineering, India
        \end{tabbing}\vspace{-18pt}
        This project was funded by DSIR, Govt. of India. I co-founded the team and worked as team co-head and technical head of this project. We developed an autonomous vehicle equipped with Stereo Vision camera, Laser range finder and GPS. 

        \section{COMPUTER SKILLS}          
        C, C++ and Matlab. \\             
\vspace{-10pt}
        \section{HONORS AND AWARDS}  
        \begin{enumerate}
        \item {\bf Gold Medal}: Best Bachelor of Engineering project award in the whole college.
        \item {\bf Best Design Award}: Awarded by Mr. Greg Henderson (President SAE International) in an event CHIMERA 06 where my team developed an amphibian robot.
        \item The only student in India who was recommended by the Ministry of HRD for the Commonwealth Scholarship Fellowship Plan, UK – 2010 for PG in Computer Science.
        %	\item Ranked 6th in an International event SNAP (Synchronized Navigation and Processing) held at Tech-Fest 2007, Asia’s largest technical fest held at IITB.
        %	\item Ranked 5th in a National level competition 'Capture the Flag' at IITM during the year 2006.

        \end{enumerate}       



        \section{PROFESSIONAL TRAININGS AND RECENT COURSES}  
        \begin{enumerate}
        \item Machine Learning Summer School 2014, Reykjavik, Iceland.
        \item Computer Vision and Machine Learning Summer School 2013, Paris, France.
        \item Some recent courses that I attended in ENS Cachan
        \begin{itemize}
        \item \emph{Deep Learning}, by Iasonas Kokkinos. 
        \item \emph{Probabilistic Graphical Models}, by Francis Bach and Guillaume Obozinski.
        \item \emph{Kernel Methods for Learning}, by Jean-Phillipe Vert.
        \item \emph{Convex Optmization}, by Alexandre d'Aspremont.
        \item \emph{Statistical Learning Theory}, by Nicolas Vayatis. 
        \item \emph{Discrete Optimization}, by N. Komodakis and M. Pawan Kumar. 
        \end{itemize}
        %\item I underwent a course on {\bf Satellite Orbits: Applied Perturbations, Maintenance and Launch} at IIT Bombay. The course was taught by  Prof. Hari Hablani.
        %\item I underwent a course on {\bf Dynamics and Control in State-Space} at IIT Bombay taught by Prof. Ashok Joshi.
        %\item I underwent a training on {\bf Space Systems} at DIAT Pune. The lectures were delivered by Prof. Brij Agarwal, Distinguished Professor, Naval Postgraduate School, USA.
        %\item I underwent a short certification course on “Real Time Software Development using POSIX”. % at one of the lab of DRDO.
        %	\item CCNA: I am a Cisco Certified Network Associate. My ID is CSCO 11473499.
        %	\item CEH: I underwent training and covered the whole curriculum of Certified Ethical Hacker v5.
        \end{enumerate}        

        \section{TEACHING}
        \begin{enumerate}
        \item I was a TA in the {\bf Coursera Course} \emph{Discrete Inference and Learning in Artificial Vision} by M. Pawan Kumar and Nikos Paragios, Jan - April 2014.
        \item Other courses that I assisted are:
        \begin{itemize}
        \item \emph{Introduction to Machine Learning}, Ecole Centrale Paris, Matthew Blaschko, 2013-14.
        \item \emph{Discrete Optimization, Ecole Centrale Paris}, M. Pawan Kumar, 2012-13.
        \item \emph{Signal Processing}, Ecole Centrale Paris, Iasonas Kokkinos, 2012-13. 
        \item \emph{Introduction to Machine Learning}, Ecole Centrale Paris, Matthew Blaschko, 2012-13.
        \end{itemize}
        \end{enumerate}


        \section{PUBLICATIONS IN JOURNALS} 
        \begin{enumerate}
        \item \emph{Rounding-based Combinatorial Algorithms for Metric Labeling}, {Under Submission in JMLR}.
        \item \emph{High Dynamic Range Fuzzy Color Image Enhancement Using Ant Colony System}, {In Journal of Applied Soft Computing}, 2012. Impact 2.97.
        \item \emph{A Novel Bacterial Foraging Technique for Edge Detection}, {In Pattern Recognition Letters}, 2011. Impact 1.46.

        %	\item {\bf\emph{A Robust Algorithm for Local Obstacle Avoidance}}, published in International Journal of Computer Theory and Engineering, Vol 2, No. 3, June 2010, 1793-8201, pp 401-404.
        \end{enumerate}  


        \section{PUBLICATIONS IN CONFERENCES (See my website for complete list)} 
        \begin{enumerate}
        \item \emph{Learning to Rank using High-Order Information}, {In ECCV 2014, Zurich}

        \item \emph{Discriminative parameter estimation for random walks segmentation}, { In MICCAI 2013, Nogoya, Japan}
        \item \emph{Learning-Based Approach for Online Lane Change Intention Prediction}, { IEEE Intelligent Vehicles Symposium (IV'2013), Australia}
        %\item {\bf\emph{Accurate Navigation of a UAV using Kalman Filter based GPS/INS Integration}}, { In SAROD 2011, Bangalore, India}
        \item \emph{A Novel Approach for Edge Detection using Ant Colony Optimization and Fuzzy Derivative Technique}, { IEEE IACC-09}, 2009, pp 1427-1433.           
        \item \emph{Probabilistic approach for Autonomous Navigation using Stereo Vision}, ICSTE 09.
        \item Technical Report \emph{“LAKSHYA The Unmanned Ground Vehicle Design Report}, Intelligent Ground Vehicle Competition (IGVC), USA, 2008, www.igvc.org/design/reports/dr238.pdf.
        %	\item {\f\emph{A Novel Software Architecture for Network Security Testing}}, INDIACom -2010.
        %	\item {\f\emph{A Robust Methodology and Architecture to Build an Unmanned System}}, got acceptance in 97th Indian Science Congress (ISC 2010).
        %	\item {\bf\emph{A Stereo Vision Algorithm for Exploration Rovers}}, INDIACom – 2008, 2nd National Conference on Computing for  Nation Development.

        \end{enumerate}

        \section{REFERENCES}    
        \begin{enumerate}
        \item Prof. M. Pawan Kumar, Ecole Centrale Paris and INRIA Saclay, \textcolor{blue}{pawan.kumar@ecp.fr}
        \item Prof. Nikos Paragios, Ecole Centrale Paris and INRIA Saclay, \textcolor{blue}{nikos.paragios@ecp.fr}
        \end{enumerate}
        \\ {\tiny -- Resume udated on {\today} --}


        \end{resume}
        \end{document}
